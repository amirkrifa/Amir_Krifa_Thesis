\chapter{Conclusions and perspectives}
\label{chapter:conclusions}

In this thesis, we have investigated the problem of mobile devices resources management towards setting up both efficient point-to-point content routing and dissemination within a delay tolerant network.
  
We first addressed the problem within the context of point-to-point DTN routing and proposed an optimal joint scheduling and buffer management policy (GBSD). We then introduced an approximation scheme (HBSD) for the required global knowledge of the optimal algorithm. Using NS2 simulations based on synthetic and real mobility traces, we showed that our policy based on statistical learning successfully approximates the performance of the optimal algorithm. Both policies (GBSD and HBSD) plugged into the Epidemic routing protocol outperform current state-of-the-art protocols like RAPID~\cite{Levine:Sigcomm07} with respect to both delivery rate and delivery delay, in all considered scenarios. Moreover, we discussed how to implement our HBSD policy in practice, by using a distributed statistics collection method, illustrating that our approach is realistic and effective. We also showed that, unlike related works~\cite{Levine:Sigcomm07, AOBM}, our statistics collection method scales well, not increasing the amount of signaling overhead during high congestion. We have also studied the distributions of HBSD utilities under different congestion levels and showed that the optimal policy heavily depends on the congestion level. The above findings suggest that methods to signal the congestion level could allow nodes to switch off the more sophisticated but \emph{heavier-duty} HBSD policy and use simpler local policies, when congestion is below some threshold. 

We then, investigated the content dissemination problem over DTN while considering the possible existence of selfish users. We have first formulated the optimal content dissemination problem in DTNs from the perspective of non-altruistic nodes while relying on a \emph{tit-for-tat} mechanism to isolate free-riders. Then, we have proposed MobiTrade, a utility-based solution to this problem that predicts the (exchange) value of each piece of content and provides a customized resource allocation strategy for each node, matched to its own interests and mobility pattern. Finally, we have showed, using a game-theoretic framework that turning on MobiTrade leads to an efficient Nash equilibrium. To our best knowledge, MobiTrade is the first content sharing
system for DTNs that can both deal with rational and selfish users while at the same time achieving good global outcomes
without explicit hard constraints on the topology and dependency of nodes or on their social behavior. Indeed, MobiTrade establishes real life trading principles by inciting users to collaborate, profiling their needs and managing their device resources optimally towards maximizing their revenues in terms of contents. Using NS3 simulations based on a synthetic mobility model (HCMM), and a real mobility trace (KAIST), we showed that selfish users are isolated and system resources are only allocated among collaborative users. 

To consolidate the NS-2/NS-3 simulations, we implemented our HBSD and MobiTrade protocols respectively as an external router for the DTN2 reference platform and as standalone mobile application for the Android powered devices. HBSD real implementation is available on our web site \cite{HBSDDTN2}, users can easily download and deploy both the DTN2 platform along with the HBSD external router and tune latter if needed. With respect to MobiTrade, the Android mobile application prototype is available for download on our web site \cite{MobiTradeAndroid}. We detail in \cite{MobiTradeAndroid} the architecture as well as the features of the MobiTrade prototype. 

As a future work, and towards consolidation our proposal (HBSD) for the point-to-point content routing problem over DTN, it would be interesting to define buffer management and scheduling policies that take into account different messages sizes (in this work, we considered that all messages have the same size). For example, in case of congestion, the end-to-end delay versus message delivery trade-off could be influenced by the choice of dropping several small messages or one large message that occupies the entire node's buffer. Then, starting from our findings which state that 
the optimal policy heavily depends on the congestion level, another interesting future work direction would be to study and design an end-to-end congestion control scheme. Indeed, signaling the congestion level could allow nodes to switch off the more sophisticated but \emph{heavier-duty} HBSD policy and use simpler local policies, when congestion is below
some threshold. We believe that such a mechanism, if available, would enable mobile nodes to save lot of resources (energy, storage and contacts' bandwidth). The latter problem remains largely not addressed in the DTN context.

Then, with respect to our proposal (MobiTrade) for the point-to-multipoint content dissemination problem over DTN, we aim at implementing the MobiTrade protocol for other types of devices and experiment with real large scale communities of users, and hence it will be possible to study MobiTrade architecture scalability under real conditions. Furthermore, it would be interesting to consider more complex content structures and their effect on MobiTrade performance. Moreover, one can go one step further with the study of the needed mechanisms to control possible advanced malicious attacks and behaviors that could impair MobiTrade content sharing sessions. Such mechanisms could be easily integrated within MobiTrade channels' utilities.
    


